\sloppy

% Настройки стиля ГОСТ 7-32
% Для начала определяем, хотим мы или нет, чтобы рисунки и таблицы нумеровались в пределах раздела, или нам нужна сквозная нумерация.
%\EqInChapter % формулы будут нумероваться в пределах раздела
%\TableInChapter % таблицы будут нумероваться в пределах раздела
%\PicInChapter % рисунки будут нумероваться в пределах раздела

% Добавляем гипертекстовое оглавление в PDF
\hypersetup{
    bookmarks=true,         % show bookmarks bar?
    unicode=true,          % non-Latin characters in Acrobat’s bookmarks
    colorlinks=true,       % false: boxed links; true: colored links
    linkcolor=black,          % color of internal links (change box color with linkbordercolor)
    citecolor=blue,        % color of links to bibliography
    anchorcolor=blue,        % color of links to bibliography
    filecolor=blue,      % color of file links
    urlcolor=cyan           % color of external links
}


% Изменение начертания шрифта --- после чего выглядит таймсоподобно.
% apt-get install scalable-cyrfonts-tex

\IfFileExists{cyrtimes.sty}
    {
        \usepackage{cyrtimespatched}
    }
    {
        % А если Times нету, то будет CM...
    }


\usepackage{G732}
\usepackage{graphicx}   % Пакет для включения рисунков

% С такими оно полями оно работает по-умолчанию:
% \RequirePackage[left=20mm,right=10mm,top=20mm,bottom=20mm,headsep=0pt]{geometry}
% Если вас тошнит от поля в 10мм --- увеличивайте до 20-ти, ну и про переплёт не забывайте:
%\geometry{right=20mm}
%\geometry{left=30mm}

\usepackage{indentfirst}
% Пакет Tikz
\usepackage{tikz}
\usetikzlibrary{arrows,positioning,shadows}

% Произвольная нумерация списков.
\usepackage{enumerate}

% ячейки в несколько строчек
\usepackage{multirow}

% itemize внутри tabular
\usepackage{paralist,array}

% Выравнивание числовых колонок
\usepackage{dcolumn}

% Центрирование подписей к плавающим окружениям
\usepackage[justification=centering]{caption}

\usepackage{listings}
\usepackage{pythonhighlight}
\usepackage{jshighlight}
% Farben definieren
\usepackage{xcolor}
\definecolor{codeGray}{RGB}{240,243,243}
\definecolor{codeBlack}{RGB}{0,0,0}
\definecolor{codeRed}{RGB}{221,0,0}
\definecolor{codeDarkRed}{RGB}{204,51,0}
\definecolor{codeDarkBlue}{RGB}{0,102,153}
\definecolor{codeBlue}{rgb}{0,153,255}
\definecolor{codeYellow}{RGB}{255,128,0}
\definecolor{codeGreen}{RGB}{0,51,51}

% … und zuweisen
\lstset{%
    language=PHP,%
    %
    % Farben, diktengleiche Schrift
    backgroundcolor={\color{codeGray}},% 
    basicstyle={\small\ttfamily\color{codeGreen}},% 
    commentstyle={\color{codeBlue}},%
    keywordstyle={\color{codeDarkRed}},%
    morekeywords={new},
    stringstyle={\color{codeRed}},%
    identifierstyle={\color{codeDarkBlue}},%
    %
    % Zeilenumbrüche aktivieren, Leerzeichen nicht hervorheben
    breaklines=true,%
    showstringspaces=false,%
    % 
    % Listing-Caption unterhalb (bottom)
    captionpos=b,%
    % 
    % Listing einrahmen
    frame=single,%
    rulecolor={\color{codeBlack}},%
    % 
    % winzige Zeilennummern links
    numbers=left,%
    numberstyle={\tiny\color{codeBlack}}%
}
\g@addto@macro\@floatboxreset\centering

